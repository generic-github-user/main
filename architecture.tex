The modern version of Eva is implemented in a functional style that maintains a single global state which is read from and written to sequentially; the command flow is completely synchronous and as such it is (at least theoretically) well-adapted for debugging and analysis.

Eva's underlying data structure would likely best be described as a "weakly typed metagraph", one of the most general possible forms of a graph data structure in which the distinction between nodes/vertices and edges is eschewed and a graph consists only of a single type of object. This atomic unit (generally referred to as a node) contains some value and an ordered (possibly empty) set of other nodes (in practice, node references). A traditional directed edge between two nodes can therefore be represented as an ordered pair of nodes, with the associated value determining their relationship. I have found this to be an empirically effective model for handling complex, dynamic data with a frequently changing schema and a requirement to be able to add additional properties to even the most basic objects.
